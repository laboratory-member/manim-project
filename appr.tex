\documentclass[12pt, a4paper, oneside]{ctexart}
\usepackage{amsmath, amsthm, amssymb, appendix, bm, graphicx, hyperref, mathrsfs}
\usepackage{braket}
\usepackage{physics}
\usepackage{subcaption}
\newenvironment{solution}
  {\par\noindent\textbf{解答:}\quad}
  {\par}

  \title{\textbf{物理中的数学近似方法:微扰论与鞍点近似}}
\author{莫登耀 \ \ 徐鸣正}
\date{\today}
\linespread{1.5}
\newtheorem{theorem}{定理}[section]
\newtheorem{definition}[theorem]{定义}
\newtheorem{lemma}[theorem]{引理}
\newtheorem{corollary}[theorem]{推论}
\newtheorem{example}[theorem]{例}
\newtheorem{proposition}[theorem]{命题}
\renewcommand{\abstractname}{\Large\textbf{摘要}}

\begin{document}

\maketitle

\setcounter{page}{0}
\maketitle
\thispagestyle{empty}

\begin{abstract}
    在理论物理与数学物理的研究中,微扰论和鞍点近似是两种重要的近似方法,广泛应用于量子力学、统计力学、场论等领域。微扰论通过将复杂问题分解为可解的零阶部分与微小扰动的高阶修正,为解析求解非线性问题提供了有效途径。鞍点近似则利用复变函数理论和最速下降法,对具有快速振荡或指数衰减特性的积分进行渐近估计。本文将重点介绍微扰论中的resolvent法和鞍点近似常规方法,并体现其在物理中的意义。
\end{abstract}

\newpage
\pagenumbering{Roman}
\setcounter{page}{1}
\tableofcontents
\newpage
\setcounter{page}{1}
\pagenumbering{arabic}

\section{数学物理近似方法的历史简述}


数学家总是诟病物理学家不够严谨。大自然并不总是那么易于合作。物理上已经有很多严格且优美的模型。但这种完全可处理的理想事件在应用数学实践中很少发生,要么是因为我们不知道确切的可解决模型,要么是因为准确的模型在理论上不可能精确解决。因此,与其使用“示范解法”,不如依赖提取近似解的方法。在这种情况下,我们可以得到足够接近的近似值,姑且称这一个问题被“解决”了吧(即使我们永远无法真正准确地解决它XD)。

从公元前200多年起,近似方法就在数学物理问题中展现出来。当时的阿基米德为了求解圆的周长,将其平均划分为许多项,并用多边形的边长和来估算圆的周长。

\begin{figure}[htbp]
    \centering
    \includegraphics[scale=0.35]{p1.png}
    \vspace{10pt} % 调整垂直间距
	\caption{阿基米德的割圆法}
\end{figure}

中世纪时期,天文学家托勒密提出的地心说模型,通过引入“本轮”和“均轮”等复杂几何结构,近似地解释了行星运动。18世纪,数学家如欧拉和拉格朗日发展了微扰理论,用于近似求解复杂的力学问题。例如,三体问题的近似解。
19世纪,统计力学的发展(如玻尔兹曼和麦克斯韦的工作)引入了概率和统计方法,用于近似描述大量粒子的行为,例如气体分子的运动。20世纪量子力学的诞生引入了波函数和算符的近似方法,用于描述微观粒子的行为在现代物理学,随着计算机技术的发展,数值模拟成为物理学中重要的近似方法,例如蒙特卡罗方法和有限元分析。

接下来我们来研究构建模型时拥有无穷小和无穷大的量时的应对方法:微扰论和鞍点近似
\newpage
\section{微扰论}
\subsection{微扰论的核心思想}
微扰理论的作用是研究小扰动系统行为的影响。例如当系统从外部接收到少量能量时,或者当实验装置受到附近街道上卡车引起的振动的轻微干扰时所添加的扰动。通常的微扰论采取以下三个步骤:

1.在已知解的模型上引入一个小的正参数$\epsilon$,称其为扰动参数。事实上我们所研究的物理对象大部分都有无穷多个这样的参数。但就像我们为了理想化模型去找去可求解的严格解那样,在选择扰动参数时考虑影响效果最明显的因素。

2.假设答案是$\epsilon$的级数形式$a=\sum_{n=0}^{\infty}a_n\epsilon^n$。这也是微扰论的核心思想:泰勒级数。事实上许多种微扰论的计算方法无非是泰勒级数思想的推广。不过计算时我们常选择截断的级数而不是无穷级数。这是因为在很多情况中微扰论无法保证所求的的无穷级数收敛,也就是说我们无法真正地逼近正确答案;但是这些级数的截断解是明显有物理意义的,让我们在一定程度上得到了答案的近似解。

3.将答案的级数形式代入模型的方程,并按照扰动参数的级次去求解。这部分的具体技术较多,下面我们从简单的例子开始引入,并介绍我们今天的主角resolvent法。

\begin{example}
    求解方程$x^5+\epsilon x-1=0$\\
\end{example}

\begin{solution}
题目中已经引入了小参数$\epsilon$,故可设答案具有$\sum_{n=0}^N x_n \epsilon^n$的形式并代入方程

\begin{equation}
    (\sum_{n=0}^N x_n\epsilon^n)^5+\epsilon(\sum_{n=0}^N x_n\epsilon^n)-1=0
\end{equation}
将扰动参数的不同级次对应起来,得到以下方程组:
\begin{equation}
    \begin{cases}
        x_0^5-1=0\\
        \epsilon(5 x_0^4x_1+ x_0)=0\\
        \epsilon^2(5x_0^4 x_2 + 10x_0^3 x_1^2 + x_1) = 0\\
        ...
    \end{cases}
\end{equation}
可求解得到
\begin{equation}
    \begin{cases}
    x_0=1\\
    x_1=-\frac{1}{5}\\
    x_2=-\frac{1}{25}\\
    ...
\end{cases}
\end{equation}
\end{solution}
\subsection{resolvent法在微扰论中的应用}
\subsubsection{resolvent算子的定义}
微扰论在量子力学中有着广泛的应用。常见的微扰论方法也和例子的思想类似。对哈密顿量引入微扰后,将本征能量和本征右矢展开成扰动参数的截断形式,再分不同的阶次去求解。为了体现微扰论和本课程的紧密联系,下面我们介绍 resolvent法在微扰论中的应用。

\begin{definition}
    对于可观测量的算符$H$,定义它的resolvent算子为$G(z)=\frac{1}{z-H}$,其中z为复变量。
\end{definition}

由于H为厄米算符,故其本征值为实数,相应G的奇点也都在实轴上。在H的本征空间内,我们有
\begin{equation}
    H_i \ket{\psi_i}=E_i\ket{\psi_i}
\end{equation}
\begin{equation}
    \langle{\psi_i}|{\psi_j}\rangle=\delta_{ij}
\end{equation}
我们将$G(z)$作用在相应的右矢上可以得到
\begin{equation}
    G(z)\ket{\psi_i}=\frac{1}{z-E_i}\ket{\psi_i}
\end{equation}
根据不同本征矢量的正交关系我们可以将$G(z)$拓展为以下形式
\begin{equation}
    G(z)=\sum_i \frac{\ket{\psi_i}\bra{\psi_i}}{z-E_i}
\end{equation}

\begin{figure}[htbp]
    \centering
    \includegraphics[scale=0.25]{p2.png}
    \vspace{10pt} % 调整垂直间距
	\caption{$G(z)$的奇点分布}
\end{figure}

实际上,\(G(z)\)包含了此量子系统的所有信息。在哈密顿算符的本征空间中,\(G(z)\)的奇点为所有本征值,且对每个单独的奇点作围道积分,可以得到相应本征矢量的投影算符。我们可以想象\(G(z)\)为一个书柜,每个格子存放着不同能级的信息,对应的围道积分操作相当于用特定的钥匙去打开格子。
\subsubsection{resolvent算子的微扰展开}
定义一个存在微扰的系统
\begin{equation}
    H=H_0+\lambda V
\end{equation}
其中$H_0$为无微扰存在时的哈密顿量,$\lambda V$为微扰量,其中$\lambda$为扰动参数。它们的resolvent算子分别为
\begin{equation}
    G(z)=\frac{1}{z-H},\qquad G_0(z)=\frac{1}{z-H_0}
\end{equation}
对于任意的算子$A,B$,我们有
\begin{equation}
    B=A+(B-A)
\end{equation}
在等式两端分别左乘$B^{-1}$,右乘$A^{-1}$,可得
\begin{equation}
    \frac{1}{A}=\frac{1}{B}+\frac{1}{B}(B-A)\frac{1}{A}
\end{equation}
令$A=z-H_0-\lambda V,B=z-H_0$代入(11)
\begin{equation}
    G(z)=G_0(z)+\lambda G_0(z)VG(z)
\end{equation}
\begin{equation}
    G(z)=\frac{G_0(z)}{1-\lambda V G_0(z)}
\end{equation}
于是可将$G(z)$展开为$\lambda$的级数
\begin{equation}
    G(z)=\sum_{n=0}^{\infty}\lambda^n G_0(z)(VG_0(z))^n
\end{equation}
\subsubsection{待求量的微扰展开}
微扰前后系统相应的本征方程为
\begin{equation}
    H_0\ket{\psi_0}=E_0\ket{\psi_0},\qquad H\ket{\psi}=E\ket{\psi}
\end{equation}
我们定义投影算子为
\begin{equation}
    P_0=\ket{\psi_0}\bra{\psi_0},\qquad  P=\ket{\psi}\bra{\psi}
\end{equation}
显然它们具有正交性
\begin{equation}
    P_i P_j=\delta _{ij}
\end{equation}
故待测量均可进行微扰展开:
\begin{equation}
    P=P_0+\sum_{n=1}^{\infty}\lambda^n \delta P^{(n)}
\end{equation}
\begin{equation}
    E=E_0+\sum_{n=1}^{\infty}\lambda^n \delta E^{(n)}
\end{equation}
根据$G(z)$的定义,我们可以用围道积分来表示$P$
\begin{equation}
    P=\frac{1}{2\pi i}\oint _{\gamma}G(z)dz
\end{equation}
围道$\gamma$的具体意义为,绕$P$所对应的本征矢$\ket{\psi}$的本征能量$E$,对应在复平面上的点$(E,0)$的极小围道。
\begin{figure}[htbp]
    \centering
    \includegraphics[scale=0.4]{p3.png}
    \vspace{10pt} % 调整垂直间距
	\caption{围道$\gamma$的走向}
\end{figure}

对于投影算子的各阶微扰可表示为
\begin{equation}
    \delta P^{(n)}=\frac{1}{2\pi i}\oint _{\gamma}G_0(V G_0)^ndz
\end{equation}
实际上投影算子的变化是波函数微扰变动的体现\\
下面我们来求能量的各阶微扰,将哈密顿算符作用于(19)两端
\begin{equation}
    HP=\frac{1}{2\pi i}\oint _{\gamma}HG(z)dz
\end{equation}
由$G(z)$的定义有$HG=zG-1$,代入上式
\begin{equation}
    HP=\frac{1}{2\pi i}\oint _{\gamma}(zG-1)dz
\end{equation}
第二项的围道积分为零,故只剩下$zG$项
\begin{equation}
    HP=\frac{1}{2\pi i}\oint _{\gamma}zG(z)dz
\end{equation}
将上式稍作变形
\begin{equation}
    HP=E_0P+\frac{1}{2\pi i}\oint _{\gamma}(z-E_0)G(z)dz
\end{equation}
取迹后我们可以得到能量的表达式
\begin{equation}
    E=Tr[HP]=E_0+Tr[\frac{1}{2\pi i}\oint _{\gamma}(z-E_0)G(z)dz]
\end{equation}
等号右端第二项可以展开成$\lambda$的级数形式
\begin{equation}
    \frac{1}{2\pi i}\oint _{\gamma}(z-E_0)G(z)dz=\sum_{n=1}^{\infty}\lambda^n \Delta^{(n)}
\end{equation}
\begin{equation}
    \Delta^{(n)}=\frac{1}{2\pi i}\oint _{\gamma}(z-E_0)G_0(VG_0)^ndz
\end{equation}
将$E$的微扰展开一一对应后,可得
\begin{equation}
    \delta E^{(n)}= Tr[\Delta^{(n)}]
\end{equation}
\subsubsection{resolvent算子的洛朗展开}
为了便于我们在(20)式和(28)式的运算,我们将$G_0(z)$进行洛朗展开。根据定义
\begin{equation}
    G_0(z)=\frac{\ket{\psi_0}\bra{\psi_0}}{z-E_0}+\sum_{i\neq 0}\frac{\ket{\psi_i}\bra{\psi_i}}{z-E_i}
\end{equation}
代入投影算子的定义
\begin{equation}
    G_0(z)=\frac{P_0}{z-E_0}+\sum_{i\neq 0}\frac{P_i}{z-E_i}
\end{equation}
其中第一项为我们所关心的本征态与本征能量,第二项求和的奇点都在围道外(见图3)。接下来将第二项展开为$(z-E_0)$的级数
\begin{equation}
    \frac{1}{z-E_i}=\sum_{k=1}^{\infty}\frac{(-1)^{k-1}(z-E_0)^{k-1}}{(E_0-E_i)^k}
\end{equation}
于是(29)式可写为
\begin{equation}
    G_0(z)=\frac{P_0}{z-E_0}+\sum_{k=1}^{\infty}(-1)^{k-1}(z-E_0)^{k-1}\sum_{i\neq 0}\frac{P_i}{(E_0-E_i)^k}
\end{equation}
此时洛朗展开的各项系数已经明朗,只需化简$\sum_{i\neq 0}\frac{P_i}{(E_0-E_i)^k}$的表达形式。根据投影算子的正交性:
\begin{equation}
    \sum_{i\neq 0}\frac{P_i}{(E_0-E_i)^k}=(\sum_{i\neq 0}\frac{P_i}{E_0-E_i})^k
\end{equation}
等号右端貌似可以表示为$G_0(E_0)$,但在(30)式中第一项,$E_0$为其一阶奇点。故可在(30)式同时左右乘$(1-P_0)$,以消去第一项
\begin{equation}
    (1-P_0)G_0(z)(1-P_0)=\sum_{i\neq 0}\frac{P_i}{z-E_i}
\end{equation}
代入(34)式,我们可以得到
\begin{equation}
    \sum_{i\neq 0}\frac{P_i}{(E_0-E_i)^k}=(1-P_0)^kG_0^k(E_0)(1-P_0)^k
\end{equation}
易知$(1-P_0)$为幂等算符,故上式简化为
\begin{equation}
    \sum_{i\neq 0}\frac{P_i}{(E_0-E_i)^k}=(1-P_0)G_0(z)^k(1-P_0)
\end{equation}
\begin{equation}
    =\frac{1-P_0}{(E_0-H_0)^k}
\end{equation}
最终我们得到了$G_0(z)$的洛朗展开形式:
\begin{equation}
    G_0(z)=\frac{P_0}{z-E_0}+\sum_{k=1}^{\infty}(-1)^{k-1}\frac{1-P_0}{(E_0-H_0)^k}(z-E_0)^{k-1}
\end{equation}
\subsubsection{代求量的具体计算}
在2.2.3中我们得到了
\begin{equation}
    \delta P^{(n)}=\frac{1}{2\pi i}\oint_{\gamma}G_0(VG_0)^ndz
\end{equation}
\begin{equation}
    \delta E^{(n)}= Tr[\Delta_n]=Tr\{\frac{1}{2\pi i}\oint _{\gamma}(z-E_0)G_0(VG_0)^ndz\}
\end{equation}
根据柯西公式,只有$(z-E_0)^{-1}$的项会对积分有所贡献。我们记洛朗展开的系数为$S_k$,于是我们展开上述两式可以得到
\begin{equation}
    \delta P^{(1)}=S_0VS_1+S_1VS_0
\end{equation}
\begin{equation}
    \delta P^{(2)}=S_0VS_1VS_1+S_1VS_0VS_1+S_1VS_1VS_0+
\end{equation}
\begin{equation}
    S_2VS_0VS_0+S_0VS_2VS_0+S_0VS_0VS_2
\end{equation}
        \qquad \qquad \qquad .....
        
\newpage
\begin{equation}
    \Delta _1=S_0VS_0
\end{equation}
\begin{equation}
    \Delta_2=S_0VS_0VS_1+S_0VS_1VS_0+S_1VS_0VS_0
\end{equation}
...

接下来我们举一个具体的例子
\begin{example}
    考虑一个二能级系统。设$H_0=\begin{pmatrix}
        E_1 & 0\\
        0 & E_2 
    \end{pmatrix},\lambda V=\lambda \begin{pmatrix}
        0&V'\\
        V'&0
    \end{pmatrix}$求出各态的波矢和能量的一,二阶微扰
\end{example}
\begin{solution}
    未微扰的状态下此系统的各本征值和本征矢为

    $\begin{cases}
        E_1\\
        E_2
    \end{cases},\qquad\begin{cases}
        \ket{1}=(1 \ 0 )^T\\
        \ket{2}=(0 \ 1 )^T
    \end{cases}$


    对1态,我们有$P_0=\ket{1}\bra{1}=diag(1 \ 0 )$故

    \(S_0=P_0=\begin{pmatrix}
        1&0\\
        0&0
    \end{pmatrix}\)
    
    \(S_1=\frac{1-P_0}{E_1-H_0}=(1-P_0)\frac{\ket{2}\bra{2}}{E_1-E_2}=\frac{1}{E_1-E_2}\begin{pmatrix}
        0&0\\
        0&1
    \end{pmatrix}\)
    
    \(S_2=-\frac{1-P_0}{(E_1-H_0)^2}=(1-P_0)(\frac{\ket{2}\bra{2}}{E_1-E_2})^2=\frac{-1}{(E_1-E_2)^2}\begin{pmatrix}
        0&0\\
        0&1
    \end{pmatrix}\)

    \(\Delta_1=S_0VS_0=0\)

    \(\Delta_2=S_0VS_0VS_1+S_0VS_1VS_0+S_1VS_0VS_0=\frac{1}{E_1-E_2}\begin{pmatrix}
        V'^2&0\\
        0&0
    \end{pmatrix}\)

    \(\delta E_1^{(1)}=Tr[\Delta_1]=0,\qquad \delta E_1^{(2)}=Tr[\Delta_2]=\frac{V'^2}{E_1-E_2}\)

    \qquad\\

    接下来计算投影算符的修正

    \(\delta P^{(1)}=S_0VS_1+S_1VS_0=\frac{1}{E_1-E_2}\begin{pmatrix}
        0&V'\\
        V'&0
    \end{pmatrix}\)

    \(\delta P^{(2)}=S_0VS_1VS_1+S_1VS_0VS_1+S_1VS_1VS_0+S_2VS_0VS_0+\)

    \(S_0VS_2VS_0+S_0VS_0VS_2=\frac{V'^2}{(E_1-E_2)^2}\begin{pmatrix}
        -1&0\\
        0&1
    \end{pmatrix}\)


    此时的初态被投影至

    $\ket{1}'=(P_0+\lambda\delta P^{(1)}+\lambda^2 \delta P^{(2)})\ket{1}=\begin{pmatrix}
        1\\
        0
    \end{pmatrix}+\frac{\lambda V'}{E_1-E_2}\begin{pmatrix}
        0\\
        1
    \end{pmatrix}-\frac{\lambda^2V'^2}{(E_1-E_2)^2}\begin{pmatrix}
        1\\
        0
    \end{pmatrix}$

    我们运用同样的过程可以得到2态的各类修正,此处不再赘述。

\end{solution}


可以发现微扰后的各态矢的近似解并非正交。但进行更多阶的微扰后,近似解会愈发逼近真实解:即新的正交态。

\subsubsection{非简并情况的讨论}
非简并的情况我们重新定义投影算符
\begin{equation}
    P_0=\sum_{i=1}^{g}\ket{\psi_{0i}}\bra{\psi_{0i}}
\end{equation}
其中g为简并度。此时在复平面上点$E_0$包含的信息为g个投影算符。微扰后新的投影算符为
\begin{equation}
    P=\sum \ket{\psi '}\bra{\psi '}
\end{equation}
此时P为g个不同投影算符的叠加。在复平面上观察的画,原先的围道只包含奇点$E_0$,故只有其解空间的子空间的相应正交子态;微扰后能级分裂,围道内包含了g个奇点,故围道积分的结果包含其中所有奇点的对应投影算符。
\begin{figure}[htbp]
    \centering
    \includegraphics[scale=0.08]{p.jpg}
    \vspace{10pt} % 调整垂直间距
	\caption{简并态的微扰前后对比}
\end{figure}

值得注意的是,简并态能级微扰中取迹的操作,得到的是平均能量的微扰,分裂能级的具体微扰还要在子空间内进行求解。
\newpage
我们接下来简略介绍非简并情况的微扰论。设简并态$E_0$的简并度为$g$,此时未微扰的投影算符为

\begin{equation}
    P_0=\sum_{\gamma=0}^{g}\ket{\psi_0,\gamma}\bra{\psi_0,\gamma}
\end{equation}
微扰后的系统可能会解除简并,也可能会保留部分简并,但我们仍将微扰后的投影算符写为
\begin{equation}
    P=\sum_{\gamma=0}^{g}\ket{\psi,\gamma}\bra{\psi,\gamma}
\end{equation}
接着我们定义将未微扰态映射到微扰态的扰动算符,它满足
\begin{equation}
    U:\ket{\psi_0,\gamma}\rightarrow\ket{\psi,\gamma}
\end{equation}
故它的形式可写为
\begin{equation}
    U=\sum \frac{\ket{\psi,\gamma}\bra{\psi_0,\gamma}}{\langle\psi,\gamma|\psi_0,\gamma\rangle}
\end{equation}
此处的未归一是为了让扰动算符作用在扰动态$\ket{\psi,\gamma}$后仍为此态。现在我们的目标是寻找算符$U$与微扰参数$\lambda$的关系,建立等式进行微扰展开。

微扰后系统相应的本征方程为
\begin{equation}
    H\ket{\psi,\gamma}=E_{\gamma}\ket{\psi,\gamma}
\end{equation}
两边减去$E_0$
\begin{equation}
    (H-E_0)\ket{\psi,\gamma }=(E_\gamma-E_0)\ket{\psi,\gamma}
\end{equation}
将未微扰的投影算符以下结果代入
\begin{equation}
    P_0(H-E_0)=\sum \ket{\psi_0,\gamma}\bra{\psi_0,\gamma}(H_0-E_0+\lambda V)=\lambda P_0 V
\end{equation}
即
\begin{equation}
    \lambda P_0 V\ket{\psi,\gamma }=(E_{\gamma}-E_0)P_0\ket{\psi,\gamma }
\end{equation}
两侧作用上微扰算符$U$
\begin{equation}
    \lambda UV\ket{\psi,\gamma }=(E_{\gamma}-E_0)U\ket{\psi,\gamma }=(E_{\gamma}-E_0)\ket{\psi,\gamma }
\end{equation}
最终我们得到了
\begin{equation}
    (E_{\gamma}-E_0-\lambda UV)\ket{\psi,\gamma }=0
\end{equation}
也可写成
\begin{equation}
    (H-E_0-\lambda UV)\ket{\psi,\gamma }=0
\end{equation}
将$\frac{\bra{\psi_0,\gamma}}{\langle\psi,\gamma|\psi_0,\gamma\rangle}$右乘至方程中,历遍$\gamma$求和
\begin{equation}
    (H-E_0-\lambda UV)U=0
\end{equation}
此处我们已经得到了投影算符关于扰动参数的方程。将$U=\sum_{n=0}^{\infty}\lambda^n U_n$代入,可得到各阶微扰的相关方程。
\newpage
\subsubsection{resolvent法和其他微扰论方法的区别}
Resolvent法和传统的微扰论(如 Rayleigh-Schrödinger 微扰论)在量子力学中都是处理复杂系统的近似方法,但 Resolvent 方法在多个方面更具有一定优势。

resolvnet法的适用范围更广。对于离散谱(束缚态)问题,两者均适用。但对于连续谱问题,只需用resolvent法将奇点略微偏离实轴便可处理。

同时其对于高阶项有着更高的计算效率,也具有更为普适的数学结构,与格林函数与路径积分也有着深刻的联系。

相较之下传统的微扰论在处理简单问题时较为直观,但在复杂问题中resolvent法更为强大与普适。

\newpage
\section{鞍点近似}
微扰论一般处理模型中关于小量的问题,而本节的方法则对参数为大量时的一种较好的渐近近似。
\subsection{鞍点近似法的理论推导}
考虑如下的复平面积分:
\begin{equation}
    I=\int_C e^{\lambda f(z)}dz
\end{equation}
我们要研究当$\lambda$趋于大量时该积分的渐近行为。通常来说我们要求$f(z)$为解析函数(有一定解析区域也可)。由其性质可知$g(z)$的实部和虚部在解析区域内无极值点,即一阶导为零的地方均为鞍点。

由于路径C与具体的路径无关,我们可以对其变形使其通过$f(z)$的鞍点$z_0$。为了简化计算,我们选择沿着最速下降的方向离开鞍点。故此时被积函数可以展开为
\begin{equation}
    e^{\lambda f(z)}\approx \exp(f(z_0)+\frac{1}{2}f''(z_0)(z-z_0)^2)
\end{equation}
\begin{figure}[htbp]
    \centering
    \includegraphics[scale=0.2]{pp.png}
    \vspace{10pt} % 调整垂直间距
	\caption{对积分路径经过鞍点的变形}
\end{figure}
即在最速下降的剖面上,我们把一个凸函数近似成高斯函数进行积分。在$\lambda$越大的情况下,误差越小。接下来我们的目标是确定最速下降的方向,再进行高斯积分。设$f''(z_0)=|f''(z_0)|e^{i\alpha},(z-z_0)=re^{i\beta}$,代入$f(z)$
\begin{equation}
    f(z)=f(z_0)+\frac{1}{2}|f''(z_0)|r^2e^{i(\alpha+2\beta)}
\end{equation}

可见当$\alpha+2\beta=(2k+1)\pi$时,高斯积分指数的实部下降最快,且虚部的贡献被消去。因此我们确定了最速下降的方向:
\begin{equation}
    \beta=-\frac{arg(f''(z_0))}{2}+\pi\pm\frac{\pi}{2}
\end{equation}
将此近似代入(49)式可得
\begin{equation}
    I=e^{i\beta}\sqrt{\frac{2\pi}{\lambda|f''(z_0)|}}e^{\lambda f(z_0)}[1+\mathcal{O}(\frac{1}{\lambda})]
\end{equation}
\subsection{鞍点近似法的实例}
\begin{example}
    求$\Gamma(t)=\int_{0}^{\infty}\rho ^{t-1}e^{-\rho}d\rho$在$t\rightarrow \infty$的渐近行为
\end{example}

\begin{solution}
    先将积分变形为标准积分形式。设$\rho=zt$,代入可得
    \begin{equation}
        \Gamma(t)=t^t \int_0^{\infty}e^{t(\ln z-z)}dz
    \end{equation}
\end{solution}
接下来寻找$f(z)=\ln z-z$在其解析区域内的鞍点
\begin{equation}
    f'(z_0)=\frac{1}{z_0}-1=0,\qquad z_0=1
\end{equation}
并确定最速下降方向
\begin{equation}
    f''(z_0)=-1,\qquad \beta= 0 \ or \ \pi
\end{equation}
\begin{figure}[htbp]
    \centering
    \begin{minipage}[b]{0.4\textwidth}
        \includegraphics[width=\textwidth]{p4.png}
        \label{fig:image1}
    \end{minipage}
    \hfill % 添加水平间距
    \begin{minipage}[b]{0.4\textwidth}
        \includegraphics[width=\textwidth]{p4(1).png}
        \label{fig:image2}
    \end{minipage}
    \caption{鞍点附近的3D图与俯视图}
    \label{fig:both_images}
    \end{figure}
\newpage
由其定义域知从0开始沿着实轴的方向刚好经过两个最速下降方向。代入公式可得
\begin{equation}
    I=\sqrt{2\pi}t^{t-\frac{1}{2}}e^{-t}
\end{equation}
此式即为斯特林近似公式
\begin{equation}
    n!\approx\sqrt{2\pi n}(n/e)^n
\end{equation}
\begin{figure}[htbp]
    \centering
    \begin{minipage}[b]{0.48\textwidth}
        \includegraphics[width=\textwidth]{p5.png}
        \label{fig:image1}
    \end{minipage}
    \hfill % 添加水平间距
    \begin{minipage}[b]{0.48\textwidth}
        \includegraphics[width=\textwidth]{p6.png}
        \label{fig:image2}
    \end{minipage}
    \caption{不同$\lambda$图像下鞍点处切面的对比图}
    \label{fig:both_images}
    \end{figure}
\begin{example}
    $H(t) = \int_{-\infty}^{\infty} \frac{e^{-t(z^2 - 1/4)} \cos t z}{1 + z^2} dz$,求其在$t\rightarrow\infty$时的渐近行为
\end{example}
\begin{solution}
    我们发现$H(t)$的被积函数在t极大时有非常明显的震荡行为,并不利于我们的计算。故先将其化为指数形式
    \begin{equation}
        H(t) = \int_{-\infty}^{\infty} \frac{e^{-t(z^2 +iz- 1/4)}}{1 + z^2} dz
    \end{equation}
    \begin{equation}
        f(z)=z^2-iz-\frac{1}{4}
        \end{equation}
    接下来寻找它的鞍点
    \begin{equation}
        f'(z_0)=-2z_0+i=0,\qquad z_0=\frac{i}{2}
    \end{equation}
    并确定最速下降方向
    \begin{equation}
        f''(z_0)=-2,\qquad \beta=0 \ or \ \pi
    \end{equation}
    .\\
    \begin{figure}[htbp]
        \centering
        \begin{minipage}[b]{0.48\textwidth}
            \includegraphics[width=\textwidth]{p7.png}
            \label{fig:image1}
        \end{minipage}
        \hfill % 添加水平间距
        \begin{minipage}[b]{0.48\textwidth}
            \includegraphics[width=\textwidth]{p8.png}
            \label{fig:image2}
        \end{minipage}
        \caption{鞍点附近的3D图与俯视图}
        \label{fig:both_images}
        \end{figure}

    这时候的形式还与鞍点近似法的标准形式有着明显的区别:除了指数函数意外,还存在$g(z)=\frac{1}{1+z^2}$的因子与其相乘。由于其相对指数函数变化更为缓慢,为了简化计算,我们可以在鞍点附近对其进行近似。
    \begin{equation}
        g(z)\approx g(\frac{i}{2})=\frac{4}{3}
    \end{equation}
    代入积分式可得
    \begin{equation}
        H(t)=\frac{4}{3}\sqrt{\frac{\pi}{t}}
    \end{equation}
\end{solution}
\begin{example}
    求Airy函数$Ai(x)=\frac{1}{2\pi}\int_{-\infty}^{\infty}e^{i(\frac{t^3}{3}+xt)}dt$在$x\rightarrow \pm\infty$的渐近行为
\end{example}
\begin{solution}
    此被积函数的定义也与鞍点近似的严格形式有所不同。但我们可以采用广义形式下的鞍点近似。即$f(z,x)=i(\frac{z^3}{3}+xz)$。虽然没有$\lambda$这类大量能提取出来,但在$x\rightarrow\infty$的情况下误差也仍在可忽略范围内。此时的积分结果表述为:
    \begin{equation}
        I=e^{i\beta}\sqrt{\frac{2\pi}{|f''(z_0,\lambda)|}}e^{f(z_0,\lambda)}
    \end{equation}
    对$x>0$的情况
    \begin{equation}
        f'(z_0,x)=i(z_0^2+x)=0,\qquad z_0=\pm i\sqrt{x}
    \end{equation}
    \begin{equation}
        f''(z_0,x)=2iz_0=\mp 2\sqrt{x},\qquad \beta = 0 \ or \ \pi
    \end{equation}

    \begin{figure}
        \centering
        \begin{subfigure}[b]{0.45\textwidth}
            \includegraphics[width=\textwidth]{p9.png}
            \caption{x>0}
            \label{fig:sub1}
        \end{subfigure}
        \hfill
        \begin{subfigure}[b]{0.45\textwidth}
            \includegraphics[width=\textwidth]{p11.png}
            \caption{x<0}
            \label{fig:sub2}
        \end{subfigure}
        \\[0.5cm] % 调整行间距
        \begin{subfigure}[b]{0.45\textwidth}
            \includegraphics[width=\textwidth]{p10.png}
            \caption{x>0}
            \label{fig:sub3}
        \end{subfigure}
        \hfill
        \begin{subfigure}[b]{0.45\textwidth}
            \includegraphics[width=\textwidth]{p12.png}
            \caption{x<0}
            \label{fig:sub4}
        \end{subfigure}
        \caption{}
        \label{fig:square}
    \end{figure}

        此时由图可注意到从$\pi$到$0$的方向,并且经过鞍点$i\sqrt{-x}$的积分路径为最速下降的路径。解得
\[
    I=\int_{i\sqrt{x}-\infty}^{i\sqrt{x}+\infty}\exp(-\frac{2}{3}x^{3/2}-\sqrt{x}(z-i\sqrt{x})^2)dz
    \sim\frac{e^{-\frac{2}{3}x^{3 /2}}}{x^{1/4}}
\]
\qquad\\
\qquad\\
对\(x<0\)的情况,
\begin{equation}
    f'(z_0,x)=i(z_0^2+x)=0,\qquad z_0=\pm \sqrt{|x|}
\end{equation}
\begin{equation}
    f''(z_0,x)=2iz_0=\pm 2i\sqrt{|x|},\qquad \beta = \frac{\pi}{4} \ or \ \frac{3\pi}{4}
\end{equation}

此时我们构造的积分路径为:沿着\(\frac{3\pi}{4}\)的方向从无穷远处经过\(-\sqrt{|x|}\)到\(-i\sqrt{|x|}\),再从\(-i\sqrt{|x|}\)沿着\(\frac{\pi}{4}\)方向经过\(\sqrt{|x|}\)到无穷远
这个时候的积分可化为\(I=I_1+I_2\),其中
\[
z+\sqrt{|x|}=e^{\frac{3}{4}\pi i}s,\qquad I_1=\int_{-\infty}^{-i\sqrt{|x|}}\exp(\frac{2i}{3}(|x|)^{3/2}-i\sqrt{|x|}(z+\sqrt{|x|})^2)dz
\]
\[
z-\sqrt{|x|}=e^{\frac{1}{4}\pi i}s,\qquad I_2=\int_{-i\sqrt{|x|}}^{\infty}\exp(-\frac{2i}{3}(|x|)^{3/2}+i\sqrt{|x|}(z-\sqrt{|x|})^2)dz
\]
在\(x\rightarrow -\infty\)的极限下,高斯函数半宽峰的量级均为\((|x|)^{-1/4}\),而对s的积分是从无穷到\(\sqrt{|2x|}\),所以\(I_1,I_2\)可看作两个完整的高斯积分。
\[
I_1=e^{\frac{3}{4}\pi i}\int_{-\infty}^{\infty}e^{\frac{2i}{3}|x|^{3/2}}e^{-|x|^{1/2}s^2}ds
\]
\[
I_2=e^{\frac{1}{4}\pi i}\int_{-\infty}^{\infty}e^{-\frac{2i}{3}|x|^{3/2}}e^{-|x|^{1/2}s^2}ds
\]
最终
\[
I=I_1+I_2=\sqrt{\frac{\pi}{|x|^{1/2}}}(e^{\frac{3}{4}\pi i}e^{\frac{2i}{3}|x|^{3/2}}+e^{\frac{1}{4}\pi i}e^{-\frac{2i}{3}|x|^{3/2}})
\]
\[
=2\sqrt{\frac{\pi}{|x|^{1/2}}}\sin(\frac{2}{3}|x|^{3/2}+\frac{3\pi}{4})
\]
所以Airy函数在x>0时的渐近行为为指数衰减;x<0时,渐近行为为震荡衰减
\end{solution}


\newpage
\subsection{鞍点近似在物理中的应用}
\subsubsection{从微正综系统到正综系统的过渡}
微正则系综描述孤立系统,固定能量 \( E \)、体积 \( V \) 和粒子数 \( N \),其熵为:
\[
S(E) = k_B \ln \Omega(E),
\]
其中 \( \Omega(E) \) 是能量 \( E \) 对应的微观状态数。

正则系综描述系统与热库接触,固定温度 \( T \)、体积 \( V \) 和粒子数 \( N \),其配分函数为:
\[
Z(T) = \int \Omega(E) e^{-\beta E} dE, \quad \beta = \frac{1}{k_B T}.
\]
将配分函数写为指数形式:
\[
Z(T) = \int e^{\frac{S(E)}{k_B} - \beta E} d E.
\]
设指数函数在能量 \( E^* \) 处取得极大值(鞍点),通过求导确定 \( E^* \):
\[
\frac{d}{dE} \left( \frac{S(E)}{k_B} - \beta E \right) = 0 \quad \Rightarrow \quad \frac{1}{k_B} \pdv{S}{E} = \beta.
\]
根据热力学关系 \( \pdv{S}{E} = \frac{1}{T} \),可得 \( \beta = \frac{1}{k_B T} \),即 \( E^* \) 对应正则系综的平衡能量。
在 \( E^* \) 附近对指数函数展开至二阶项:
\[
\frac{S(E)}{k_B} - \beta E \approx \frac{S(E^*)}{k_B} - \beta E^* + \frac{1}{2} \left( E - E^* \right)^2 \pdv[2]{E} \left( \frac{S(E)}{k_B} - \beta E \right).
\]
计算二阶导数:
\[
\pdv[2]{S}{E} = -\frac{1}{k_B T^2 C_V} \quad \Rightarrow \quad \text{负定(保证鞍点为极大值)}.
\]
代入高斯积分近似:
\[
Z(T) \approx e^{\frac{S(E^*)}{k_B} - \beta E^*} \sqrt{\frac{2\pi k_B T^2 C_V}{N}}.
\]
亥姆霍兹自由能 \( F = -k_B T \ln Z(T) \),取热力学极限(\( N \to \infty \),涨落可忽略):
\[
F \approx E^* - T S(E^*).
\]
此时 \( E^* \) 为平衡能量,对应正则系综的平均能量 \( \langle E \rangle \)。

\subsubsection{鞍点近似在Ising模型中的应用}
考虑无限范围Ising模型,其哈密顿量为:
\[
H = -\frac{J}{2N} \sum_{i \neq j} S_i S_j - h \sum_{i=1}^N S_i
\]
配分函数为:
\[
Z = \sum_{\{S_i\}} e^{-\beta H}
\]
其中 \( \beta = 1/(k_B T) \)。
经过一系列数学处理后,配分函数变为:
\[
Z = e^{\frac{\beta J}{2}} \sqrt{\frac{\beta J}{2\pi N}} \int_{-\infty}^\infty dm \, e^{-N \left[ \frac{\beta J}{2} m^2 - \ln \left( 2 \cosh(\beta (J m + h)) \right) \right]}
\]
其中\(m\)为磁化强度,\(h\)为外磁场,\(J\)为铁磁耦合强度。
在热力学极限 \( N \to \infty \) 下,积分由指数中被积函数的极小值(鞍点)主导。
令有效自由能密度为:
\[
f(m) = \frac{\beta J}{2} m^2 - \ln \left( 2 \cosh(\beta (J m + h)) \right)
\]
配分函数近似为:
\[
Z \propto \int dm \, e^{-N f(m)}
\]
鞍点 \( m_0 \) 满足极值条件:
\[
\left. \frac{df}{dm} \right|_{m = m_0} = 0
\]
计算导数:
\[
\frac{df}{dm} = \beta J m_0 - \beta J \tanh\left( \beta (J m_0 + h) \right) = 0
\]
化简得自洽方程
\[
m_0 = \tanh\left( \beta (J m_0 + h) \right)
\]
这正是平均场理论的结果,对应磁化强度 \( m \)。

在鞍点处,自由能密度为:
\[
f(m_0) = \frac{\beta J}{2} m_0^2 - \ln \left( 2 \cosh(\beta (J m_0 + h)) \right)
\]
在鞍点附近展开 \( f(m) \) 至二阶项:
\[
f(m) \approx f(m_0) + \frac{1}{2} f''(m_0) (m - m_0)^2
\]
其中二阶导数为:
\[
f''(m_0) = \beta J \left[ 1 - \beta J \cdot \text{sech}^2\left( \beta (J m_0 + h) \right) \right]
\]

配分函数近似为:
\[
Z \propto e^{-N f(m_0)} \int_{-\infty}^\infty d\eta \, e^{-\frac{N}{2} f''(m_0) \eta^2}
\]
其中 \( \eta = m - m_0 \)。高斯积分结果为:
\[
Z \propto e^{-N f(m_0)} \sqrt{\frac{2\pi}{N f''(m_0)}}
\]

总自由能密度为:
\[
F = -\frac{1}{\beta N} \ln Z \approx f(m_0) - \frac{1}{2\beta N} \ln \left( \frac{2\pi}{N f''(m_0)} \right)
\]
在热力学极限下,第二项趋于零,自由能由鞍点主导:
\[
F \approx f(m_0)
\]

\subsubsection{从量子到经典的过渡}

量子系统的传播子表示为所有可能路径的积分:
\[
K(x_f, t_f; x_i, t_i) = \int \mathcal{D}[x(t)] \, e^{i S[x(t)]/\hbar}
\]
其中作用量 \( S[x(t)] = \int_{t_i}^{t_f} L(x, \dot{x}, t) \, dt \),\(\mathcal{D}[x(t)]\) 为路径积分测度。

当 \(\hbar \to 0\) 时,指数项 \( e^{i S/\hbar} \) 的相位剧烈振荡,主要贡献来自作用量的极值路径(鞍点)。

对作用量 \( S[x(t)] \) 做变分(即最小作用量原理),得到欧拉-拉格朗日方程:
    \[
    \frac{\delta S}{\delta x(t)} = 0 \quad \Rightarrow \quad \frac{d}{dt} \left( \frac{\partial L}{\partial \dot{x}} \right) - \frac{\partial L}{\partial x} = 0
    \]
    满足此方程的路径 \( x_{\text{cl}}(t) \) 称为经典路径。
将路径分解为经典路径与量子涨落:
\[
x(t) = x_{\text{cl}}(t) + \delta x(t), \quad \delta x(t_i) = \delta x(t_f) = 0
\]
将作用量展开至二阶项:
\[
S[x(t)] = S[x_{\text{cl}}] + \frac{1}{2} \iint \frac{\delta^2 S}{\delta x(t) \delta x(t')} \delta x(t) \delta x(t') \, dt dt' + \mathcal{O}(\delta x^3)
\]

传播子近似为:
\[
K \approx e^{i S[x_{\text{cl}}]/\hbar} \cdot \int \mathcal{D}[\delta x(t)] \, \exp\left( \frac{i}{2\hbar} \iint \frac{\delta^2 S}{\delta x(t)\delta x(t')} \delta x(t) \delta x(t') \, dt dt' \right)
\]
高斯积分结果为:
\[
K \propto e^{i S[x_{\text{cl}}]/\hbar} \cdot \left( \det \left( -\frac{\delta^2 S}{\delta x^2} \right) \right)^{-1/2}
\]
当 \(\hbar \to 0\) 时,量子涨落项的贡献会减小,系统的行为退化回经典的过程。

\qquad\\
通过以上三个例子可以看出,鞍点近似是量子世界和经典世界联系起来的桥梁。鞍点处一阶导为零的条件通常直接对应经典运动方程(如欧拉-拉格朗日方程)或宏观物理量的极值条件;鞍点处展开二阶导项的积分近似通常为微观/量子情况下的最低修正。
\newpage

\begin{thebibliography}{99}
    \bibitem{a}Nicolas Fillion, Robert M. Corless. \emph{Perturbation Theory}[M]. 14 Dec 2022.
    \bibitem{b}Daniel Adam Steck. \emph{Quantum Mechanics}.  0.2.17, 13 November 2024.
    \bibitem{c} George B. Arfken. \emph{Mathmatical Methods For Physicists}.   seventh edition.
    \bibitem{d}Angel Alastuey. \emph{Phisics and Mathematical Tools}. 2016.
    \bibitem{e}https://zhuanlan.zhihu.com/p/68639143
\end{thebibliography}



\end{document}